\section{Introduction}\label{sec:introduction-\myInitials}

% Provide an introduction to the problem area(s) for which the kernel you want to study is relevant. For this chapter, you DO NOT need to discuss why graphs are important in general, what graphs are, or other ``common'' things like how to represent graphs. Such things will be addressed in a common part of the Compendium report. Clearly, though, when you convert your chapter into a paper you will have to add a bit of such stuff back into the intro, and perhaps elsewhere.

In 2014 the world saw the most massive resurgence of the Ebola virus in history. The epidemic was mainly localized to West Africa but spread to other parts of Africa, and even other countries through a myriad of transportation methods. All told, nearly 30,000 people were infected worldwide, of which 11,300 died. These numbers are terrible, but they could have easily been worse. A large part of planning against the transmission of epidemics is in modeling the dispersal of infection through physical travel networks. In 2014 this modeling helped the Centers for Disease Control (CDC) increase monitoring on select ports to slow or prohibit the possibility of widespread transmission. The goal of this paper is to investigate models of disease transmission that utilize Random Walks as their underlying kernel. 