\section{Some Realistic Data Sets}\label{sec:basic-datasets-\myInitials}

\subsection{Example Data Sets}

\begin{table}[h]
    \centering
    \caption{Real-world datasets and their sizes.}
    \\
    \begin{tabular}{lll}
        % \multicolumn{3}{c}{{\color[HTML]{343434} \textbf{Potential Datasets}}}\\
        \hline
        \multicolumn{1}{|l|}{\textbf{Dataset}}  & \multicolumn{1}{l|}{\textbf{Approx Vertices}} & \multicolumn{1}{l|}{\textbf{Approx Edges}}   \\ \hline
        \multicolumn{1}{|l|}{DOT Railway Data}  & \multicolumn{1}{r|}{196K}           & \multicolumn{1}{r|}{250K} \\ \hline
        \multicolumn{1}{|l|}{SNAP Airport Data} & \multicolumn{1}{r|}{456}            & \multicolumn{1}{r|}{71K}       \\ \hline
        \multicolumn{1}{|l|}{KNB Shipping Data} & \multicolumn{1}{r|}{3700}     & \multicolumn{1}{r|}{15K}       \\ \hline
    \end{tabular}
    \label{table:potential_datasets}
\end{table}

Transportation networks, in general, are pretty ubiquitous. The SNAP Labs\footnote{\cite{snapnets}https://snap.stanford.edu/data/reachability.html} usually maintain stable data repositories for network data, and here is no exception. The \textbf{Airline Travel Reachability Network} contains both US and Canadian airport travel nodes and edges, including plenty of metadata such as travel time, geodata, and other potentially useful things. 

Maritime data is more difficult to find, but there are multiple years worth of shipping data maintained by The Knowledge Network for Biocomplexity (KNB)\cite{knb_data}, and increasing amounts of data in this space is becoming open access. Railway datasets are less convenient as the datasets are generally maintained by the respective governments for which the railway belongs. For instance, US railway data can be found at either Data.gov, or the Department Of Transportation data page\footnote{osav-usdot.opendata.arcgis.com/}.The size and order of these datasets can be found in table \ref{table:potential_datasets}.

\subsection{Constructing A Complex Network}

If we were to construct a complex network from the above transportation network datasets, we would find that the size and order of the network would increase greatly. For instance, if an airport terminal also contains a train terminal (not irregular) then the airport and train station nodes must either merge, or copy the inbound and outbound edges of their counterpart. This sort of node merging or edge duplication activity results in quite irregular networks. This irregularity, coupled with the network's heterogeneity, makes it difficult to build network generators that sufficiently capture the depth of interactions and metadata represented in real-world data.

Moreover, as stated above, many of the datasets are maintained by separate entities and represent transportation networks that serve different goals. For instance, the difference between passenger trains and freight trains means a great deal when attempting to capture human vectored disease transmission. For these reasons, interested parties such as the World Health Organization (WHO) and the CDC are working with countries all over the world to increase the accuracy of and access to this data.

\subsection{Generating Representative Datasets}
Given the difficulties discussed above related to wrangling the real-world datasets, we chose to use synthetically generated datasets of several types. We used several sizes of grid networks where vertices were given random initial populations, the edges outgoing from a vertex is set as a proportion of the vertex population and out degree. A vertex is then selected at random and some small portion of their population is infected initially.

When using graphs that are not grid structured, Erdős–Rényi for example, because the vertex degrees is irregular we use them to inform the vertex population size. Greater degree vertices have greater population sizes. A vertex is still selected to have an initial infected population, but instead of random uniform sampling, the sampling is weighted by vertex population. Higher population vertices are more biased towards having the initial observed infective population. 

We plan to investigate other real-word adjacent graph generation techniques in further work.